\frame{Factorizaci\'on}

\frame{
\frametitle{Factorizacion}
\framesubtitle{Reducci\'on a order finding}

  Sea $n \in \mathbb{N} $, con $N = pq$, $p$ y $q$ primos "grandes".
  \pause
  
  Consideramos la ecuaci\'on:
  $$x^2 \equiv 1\mod N$$
  \pause
  
  Es equivalente a
  $$N \vert \left( x+1 \right) \left( x-1 \right)$$
  \pause

  Si $x \not\equiv \pm 1$ entonces $p \vert ( x+1 ) $ y $q \vert ( x-1 )$
  (s.p.d.g.).
  \pause
  
  De hecho $p = gcd(x+1,N)$, $q=gcd(x-1, N)$
}  


\section{Order finding a Period Finding}
\frame{\frametitle{Factorizaci\'on}
\framesubtitle{Order Finding}
Sea $x\in\mathbb{N}$, con $m \perp N$. \pause

Si existe una forma eficiente
de calcular $O(m)$, y es par, entonces tomamos $x = m^{\frac{O(m)}{2}}$.
\pause

Tomando un $m$ arbitrario,
?`Cu\'al es la probabilidad de que $O(m)$ sea par y $x$ no trivial?\pause

\huge{ALTA}
}

\frame{\frametitle{Order Finding}
\framesubtitle{Period finding}
Encontrar el \'orden de $m$ es an\'alogo a encontrar el per\'iodo de
$f(n) = m^n$
}




